% Generated on 2023/10/16 15:02:38
\documentclass[12pt,a4paper]{report}
\usepackage[utf8]{inputenc}
\usepackage[explicit]{titlesec}
\usepackage[dvipsnames]{xcolor}
\usepackage{amsmath}
\usepackage{amsfonts}
\usepackage{url}
\usepackage{amssymb}
\usepackage{makeidx}
\usepackage{graphicx}
\usepackage{tikz}
\usepackage[many]{tcolorbox}
\usepackage{standalone}
\usepackage{animate}
\tcbuselibrary{listings}
\usepackage{tabularx}
\usepackage{colortbl}
\usepackage{multicol}
\usepackage{makecell}
\usepackage{adjustbox}
\usepackage{subfig}
\usepackage{lipsum}

\usepackage{framed}
\usepackage{lastpage}


\usepackage{fancyhdr}
\usetikzlibrary{calc,arrows.meta,backgrounds}

\newcommand{\msymbol}[1]{\ifmmode #1 \else $#1$\fi}

\newcommand{\daemathvariable}[1]{{\normalsize \color[RGB]{100,149,237}#1}}
\newcommand{\daemathvalue}[1]{{\normalsize \color[RGB]{153,153,255}#1}}
\newcommand{\daemath}[1]{{\large \color[RGB]{0,0,0}#1}}
\newcommand{\daered}[1]{{\normalsize \color[RGB]{234,0,33}#1}}
\newcommand{\daegreen}[1]{{\normalsize \color[RGB]{0,255,65}#1}}
\newcommand{\mmathvariable}[1]{{\begingroup  \color[RGB]{100,149,237}#1 \endgroup} }
\newcommand{\mmathvalue}[1]{{\begingroup  \color[RGB]{153,153,255}#1 \endgroup} }
\newcommand{\mmath}[1]{{\begingroup  \color[RGB]{0,0,0}#1 \endgroup} }
\newcommand{\mred}[1]{{\begingroup  \color[RGB]{234,0,33}#1 \endgroup} }
\newcommand{\mgreen}[1]{{\begingroup  \color[RGB]{0,255,65}#1 \endgroup} }
\definecolor{backgroundStroke}{RGB}{0,0,0}
\definecolor{backgroundFill}{RGB}{255,255,255}
\definecolor{backgroundText}{RGB}{0,0,0}
\definecolor{bgChapter}{RGB}{91,155,213}
\definecolor{bgSection}{RGB}{222,234,246}
\definecolor{fontSubsection}{RGB}{53,124,177}
\colorlet{shadecolor}{bgChapter}

\renewcommand{\contentsname}{Contents}
\newcommand{\pbreak}{\vskip 0.5cm}
\renewcommand{\thesection}{\arabic{section}}

\newcommand{\daechapter}[1]{
  \chapter*{#1} % Create an unnumbered chapter
  \addcontentsline{toc}{chapter}{#1} % Add the chapter to the ToC
}

\newcommand{\daesection}[1]{
  \section*{#1} % Create an unnumbered chapter
  \addcontentsline{toc}{chapter}{#1} % Add the chapter to the ToC
}

\newtcolorbox{ybox}{
  colback=yellow,
  colframe=yellow,
  boxrule=0pt,
  arc=0pt,
  outer arc=0pt,
  boxsep=0pt,
  left=1pt,
  right=1pt,
  top=1pt,
  bottom=1pt
}

\newtcolorbox{gbox}{
  colback=lightgray,
  colframe=lightgray,
  boxrule=0pt,
  arc=0pt,
  outer arc=0pt,
  boxsep=0pt,
  left=1pt,
  right=1pt,
  top=1pt,
  bottom=1pt
}



\pagecolor{backgroundFill}
\color{backgroundText}
\everymath{\color{backgroundText}}

\titleformat{\chapter}[display]{\normalfont\color{white} \begin {shaded*}\bfseries}{\large\chaptername~\thechapter}{20pt}{\Large#1\end{shaded*}}

\titleformat{\section}
  {\normalfont\Large\bfseries} 
  {}{0em} 
  { 
    \colorbox{bgSection}{%
      \parbox{\dimexpr\textwidth-2\fboxsep\relax}{\thesection\quad#1}
    }
  }
  
\titleformat{\subsection}
  {\color{fontSubsection}\titlerule\vspace{1ex} \normalfont\large\bfseries}
  {\thesubsection}{1em}
  {
  	\quad #1
  }
  
\titleformat{\subsubsection}
  {\color{fontSubsection}\titlerule\vspace{1ex}\normalfont\normalsize\bfseries}
  {\thesubsubsection}{1em}
  {	
  	 \quad #1   	 
  }
 
% information about the gradwork
\def\title{Incremental Performance and Quality Analysis of Hybrid Ray Tracing: Vulkan vs DXR in Real-Time Rendering}
\def\subtitle{Subtitle}
\def\author{De Meyer Luca}
\def\academicyear{2025-2026}


\fancyhf{}
\fancyhead[L]{\author}
\fancyfoot[L]{DAE - Graduation work 2025-2026}
\fancyfoot[R]{\thepage/\pageref{LastPage}}


\begin{document}
\pagestyle{fancy}




\begin{titlepage}
   \begin{center}
       \vspace*{1cm}
       \Huge{\title}
       \vspace{0.5cm}
       
       \Large{\subtitle}
       \vspace{1.5cm}
       
       \textbf{\author}
       \vfill
       \Large{Graduation work \academicyear}
      
     
		\begin{figure}[h]
    		\begin{minipage}[t]{0.40\textwidth}
        		\includegraphics[height=2cm]{logos/Howest_logo.png}
    		\end{minipage}\hfill
	    	\begin{minipage}[t]{0.40\textwidth}
    		    \centering
		        \includegraphics[height=2cm,trim=2cm 7cm 0.5cm 6cm,clip]{logos/DAE_logo.pdf}
       
    		\end{minipage}
			\vspace{0.8cm}
		\end{figure}
   \end{center}
\end{titlepage}

\tableofcontents

\daechapter{Abstract {\&} Keywords}
\begin{ybox}
Real-time rendering has converged on hybrid pipelines that combine rasterization with selectively applied ray tracing. Effects such as shadows, reflections, and global illumination improve visual fidelity, but their performance costs vary widely across scenes and sampling strategies.

This study evaluates incremental performance and visual impact of ray-traced shadows and reflections using Vulkan Ray Tracing and DXR. Multiple hybrid strategies (G-buffer-guided ray generation, adaptive sampling, screen-space fallbacks) are analyzed at 1080p and 1440p. Visual quality is assessed against a path-traced reference using LPIPS, SSIM, while GPU profiling measures feature-level costs.

We derive a quality-per-millisecond metric enabling direct comparison of hybrid rendering configurations, providing practical guidance for incremental ray tracing adoption.
\end{ybox}

\begin{gbox}
Keywords: Real-time rendering, Hybrid Ray Tracing, Vulkan Ray Tracing, DXR, Shadows, Reflections, Performance Analysis, Perceptual Quality
\end{gbox}

\daechapter{Preface}
\begin{ybox}
This graduation project explores incremental hybrid ray tracing strategies in real-time rendering. The goal is to provide developers practical guidance on prioritizing ray-traced features under fixed frame-time budgets.
\end{ybox}

\begin{gbox}
I would like to thank my supervisors and peers at Howest-Digital Arts and Entertainment for their guidance and feedback throughout the research process.
\end{gbox}


daechapter{List of figures}
\addcontentsline{toc}{chapter}{List of figures}

\begin{ybox}
Figure 1: Graphics Rendering Pipeline
\end{ybox}
\begin{ybox}
Figure 2: Geometry Processing Pipeline
\end{ybox}
\begin{ybox}
Figure 3: Rasterization Stage
\end{ybox}
\begin{ybox}
Figure 4: Sample Hybrid Ray Tracing Diagram
\end{ybox}

\daechapter{Introduction}
\addcontentsline{toc}{chapter}{Introduction}

\daechapter{Introduction}
\begin{ybox}
Hardware-accelerated ray tracing is now standard in consumer GPUs, typically deployed incrementally in hybrid pipelines. Fully path-traced rendering remains expensive, so ray tracing is applied selectively for shadows and reflections, while rasterization handles primary shading. Allocating ray budgets efficiently is challenging due to scene complexity, sampling strategy, and perceptual benefits.
\end{ybox}

\begin{gbox}
Research questions:
\begin{enumerate}
    \item What are the performance costs of ray-traced shadows and reflections individually and combined?
    \item How do these costs scale with resolution and scene complexity?
    \item Which hybrid strategies maximize quality-per-millisecond?
    \item Do Vulkan and DXR perform similarly on identical workloads?
\end{enumerate}

Hypotheses:
\begin{itemize}
    \item H1: Shadows are cheaper than reflections; combined cost is not strictly additive.
    \item H2: 1 SPP + temporal denoising offers best quality-per-millisecond.
    \item H3: Vulkan and DXR perform within ±10\%.
\end{itemize}
\end{gbox}

\begin{gbox}
\msymbol{\alpha}, \msymbol{\beta}, \msymbol{\gamma} are used for denoting angles in reflection computations.
\end{gbox}

\begin{gbox}
Math symbols, method to use math symbol inside or outside of math text with the msymbol command:
\msymbol{\alpha},\msymbol{\beta}
\end{gbox}

\daechapter{Literature study / Theoretical framework}

\begin{ybox}
Rasterization and ray tracing fundamentals, hybrid strategies, and perceptual metrics (LPIPS, SSIM, PSNR) are introduced. Previous work on Vulkan vs DXR performance comparisons is reviewed. Hybrid strategies include full-screen, G-buffer-guided, screen-space fallback, and adaptive sampling.
\end{ybox}

\begin{gbox}
Hybrid pipelines balance performance and visual fidelity by selectively tracing rays where screen-space or rasterization approximations fail.
\end{gbox}

\daechapter{Research}

\begin{ybox}
This section presents experimental methodology: test scenes, independent variables, hardware setup, instrumentation, and evaluation metrics.
\end{ybox}

\begin{gbox}
Three test scenes: Indoor, Outdoor, Specular. Variables: resolution, ray-traced features, SPP, denoiser, hybrid strategy, trace resolution, light count, scene complexity. Hardware: RTX 3060 Mobile, RTX 4090 Mobile, RX 9070 XT.
\end{gbox}


\clearpage
\section{Experimental Design}

\subsection{Data Collection}

\begin{gbox}
GPU timestamps for each pipeline stage: rasterization, shadow RT, reflection RT, denoising, post-processing. Each configuration runs 120 frames; middle 60 frames analyzed for stability.
\end{gbox}

\subsubsection{Quality Evaluation}

\begin{gbox}
Metrics: LPIPS, SSIM, PSNR, temporal LPIPS differences for flicker/ghosting. A/B visual tests validate metrics. Quality-per-millisecond metric computed as:
\[
\text{QPM} = \frac{1 - \frac{\text{LPIPS}_{\text{config}}}{\text{LPIPS}_{\text{raster}}}}{\text{RT Overhead (ms)}}
\]
\end{gbox}

\section{Results}

\subsection{Performance Costs}

\begin{gbox}
Expected: Shadows ~2ms, Reflections 4–6ms, Combined 7–9ms at 1080p. 1080p→1440p scaling ~1.8×. Vulkan and DXR ±5–10\%.
\end{gbox}

\subsection{Quality Optimization}

\begin{gbox}
1 SPP + temporal denoising achieves LPIPS within 10\% of 4 SPP raw while costing 75\% less. Adaptive sampling reduces cost 40–60\% with <5\% quality loss. G-buffer-guided tracing provides best QPM.
\end{gbox}


\daechapter{Case Study}

\begin{ybox}
Detailed evaluation of hybrid strategies in three test scenes, comparing Vulkan vs DXR. Tables and figures illustrate frame time breakdown, ray counts, and perceptual quality metrics.
\end{ybox}

\section{Introduction}
\begin{ybox}
Background, objectives, and research questions of the case study. Incremental hybrid ray tracing evaluated in real-time scenarios.
\end{ybox}


\section{Modelling}

\subsection{Blockout}

\begin{gbox}
\lipsum[1-1]
\end{gbox}

\begin{figure}[h]
  \centering
    
      \includegraphics[trim=2cm 7cm 0.5cm 6cm,width=1.0\linewidth]{logos/DAE_logo.pdf}
  \caption{The dae logo}
\end{figure}

\begin{gbox}
\lipsum[1-1]
\end{gbox}

\subsection{ZBrush}

\begin{gbox}
\lipsum[1-1]
\end{gbox}

\section{Texturing}

\begin{gbox}
\lipsum[1-1]
\end{gbox}

\section{Shading}

\begin{gbox}
\lipsum[1-1]
\end{gbox}

\section{Lighting}

\begin{gbox}
\lipsum[1-1]
\end{gbox}

\daechapter{Discussion}
\begin{ybox}
Interpretation of results in the context of hypotheses. G-buffer-guided tracing with temporal denoising consistently yields highest QPM. Vulkan and DXR perform similarly. Scene complexity affects reflections more than shadows.
\end{ybox}

\begin{gbox}
Observations: Adaptive sampling reduces cost with minimal quality loss. Full-screen ray tracing is less efficient. Temporal denoising improves perceptual similarity significantly.
\end{gbox}

\daechapter{Conclusion}
\begin{ybox}
Incremental hybrid ray tracing for shadows and reflections can improve perceptual quality efficiently. G-buffer-guided + temporal denoising offers best quality-per-millisecond. Vulkan and DXR exhibit near-identical performance.
\end{ybox}


\daechapter{Future work}
\begin{ybox}
Extend analysis to global illumination and ambient occlusion, investigate ML-based denoisers, dynamic quality scaling, and next-gen GPU performance.
\end{ybox}

\begin{gbox}
\lipsum[1-1]
\end{gbox}

\daechapter{Critical Reflection}
\begin{ybox}
The project enhanced knowledge in GPU ray tracing, hybrid rendering strategies, performance analysis, and perceptual quality evaluation. Learned integration of Vulkan and DXR in real-time engines.
\end{ybox}


\daechapter{References}
\begin{ybox}
Zhang et al., LPIPS: Learned Perceptual Image Patch Similarity, 2018.\\
Unreal Engine 5 Documentation, Offline Path Tracing.\\
Casey Raes (2014), Processing (second edition).\\
Sarah Northway (2016) A Year in VR, GDC Vault.
\end{ybox}
\end{document}
